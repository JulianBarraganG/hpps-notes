\chapter{Data as Bytes}
\label{chap:bytes}

In the previous chapters we looked at data represented as bits.  While
fiddling with bits is indeed how computation is ultimately carried out
in a physical sense, computers are built as layers of aggregation and
abstraction.  We already saw how individual bits were assembled into
bit vectors, and how bit vectors of known lengths can be used to
represent data types such as numbers.  In principle, we could imagine
a computer that simply treated the totality of available data as a
single large bit vector.  In practice, the smallest storable unit of
data is a bit vector called a \emph{byte}, which is stored in
\emph{memory}, and which is the subject of this chapter.

The size of a used to vary between computers, and tended to be the
amount of bits necessary to represent a single character.  Many older
computers used 6-bit bytes, as $2^{6}=64$ allows enough distinct
values to represent the alphabet, ten digits, and some punctuation
characters\footnote{We will return to the issue of text representation
  in \cref{sec:text}}.  However, essentially all modern computers use
8-bit bytes.  In fact, ``byte'' has become more or less synonymous
with ``8 bits'', although a pedant would insist that the proper term
for such a bit vector is \emph{octet}.  For simplicity, we will use
``byte'' in this text.

As we saw earlier, many data types are represented in more than 8
bits.  To represent such data, we need multiple bytes.  Because the
byte unit is the unit of storage, all common data types are a multiple
of 8 bits, meaning one or more whole bytes.  Working with data that is
not expressible as a whole number of bytes is possible, but is more of
a chore.

When discussing values at the byte level, hexadecimal (base-16)
notation is often used, where the letters $a$--$f$ (or their uppercase
equivalents) are used to denote digits with values $10$--$15$.
Hexadecimal notation is convenient because the 16 different digits for
a single hexadecimal digit corresponds exactly to 4 bits, and a single
8-bit byte can thus be written using exactly two hexadecimal digits.
\begin{equation}
  149_{10} = 10010101_{2} = 95_{16}
\end{equation}
It is usually easier to read a hexadecimal number than a long binary
number. Also, it is easy to translate hexadecimal numbers to binary,
because we can translate each digit separately and simply concatenate
at the end.

\section{Memory}
\label{sec:memory}

The main working storage of a computer is called the \emph{memory}.
Conceptually, the memory can be seen as a large contiguous array of
bytes.  Given an index, called an \emph{address}, we can retrieve or
modify the byte stored at that address.  \Cref{fig:memory} shows this
conceptual model.  This kind of memory is called \emph{random access
  memory} (or RAM) because we can at any time access the data at any
address.  When a program is running, its data (values of variables and
objects) is stored in memory.

\begin{figure}
  \centering
  \begin{tabular}{l|c|c|c|c|c|c|c|c|c|c|c|c|r}
    \multicolumn{1}{l}{$00\cdots{}0_{16}$}
    &\multicolumn{12}{c}{}
    &\multicolumn{1}{l}{$FF\cdots{}F_{16}$} \\
    \cline{2-13} &&&&&&$\cdots$&&&&&&& \\\cline{2-13}
  \end{tabular}
  \caption{Memory is an array of bytes, indexed from $0$ to some
    largest address, depending on the size of the memory.  Addresses
    are typically written with hexadecimal (base-16) numbers.}
  \label{fig:memory}
\end{figure}

Physically, memory is usually implemented using \emph{dynamic RAM}
(DRAM) technology, although other technologies have been used in the
past.  DRAM needs to be constantly refreshed, requiring power, or it
will lose data.  This means that data stored in DRAM will be lost when
the computer is turned off.  Data that we intend to keep must be
written to persistent storage, which today typically takes the form of
a hard disk drive (HDD) or solid state drive (SSD), although many
other technologies exist.  These storage technologies are not
typically considered part of ``memory'', but are made available
through \emph{file systems}, although the line is somewhat blurred by
the notion of \emph{virtual memory} where data is transparently moved
between different physical storage media (outside the scope of this
chapter).

In principle, to perform a computation such as addition, the processor
must fetch the operands from memory, perform the operation as
described in previous chapters, and write the result back to memory.
In practice, processors contain a fixed set of \emph{registers} that
we can view as a small collection of fixed-size variables.  The
processor operates on data stored in registers, and copies explicitly
between registers and memory as needed.  As programmers we can usually
ignore this distinction---it is dealt with by the implementation of
the programming language we are using.  We will return to this in
\cref{chap:compiled-interpreted}.

\section{Multi-byte Words}

When we store a multi-byte object in memory, such as a 64-bit
floating-point number, we identify it with the address of its
\emph{first} constituent byte, counting from low addresses to high
addresses.  As always, data is not self-describing, so it is in
principle our responsibility to remember the number of constituent
bytes.  For simple data types, the size is known.  For example, in C
on a modern x86-64 machine, the \texttt{char} type is always 1 byte, a
\texttt{short} is 2 bytes, and an \texttt{int} is 4 bytes.  We will
discuss variable-size data types in \cref{sec:text} and
\cref{chap:datalayout}.

\subsection{Byte Ordering}
\label{sec:endianness}

When storing a multi-byte object (say, a 4-byte \texttt{int}) in
memory, each individual byte will have an address.  This means that
the \emph{byte order} is visible at the memory level, and hence that
the order in which we store the bytes matter.  This is called
\emph{endianness}, a term taken from \emph{Gulliver's Travels} where
the Lilliputians fight a civil war over whether the shell of a boiled
egg should be broken from the big or little end.

In computer science, there are two main endianness conventions:
\begin{definition}[Big Endian]
  The most significant byte is stored at the lowest address.
\end{definition}
\begin{definition}[Little Endian]
  The least significant byte is stored at the lowest address.
\end{definition}

\begin{example}[Representing $x=01234567_{16}$ in memory]
  With big-endian, the byte order would be $01~23~45~67$, while with
  little-endian the order would be $67~45~23~01$.  If $x$ is stored
  starting at address $p$, then address $p$ would contain the byte
  $01$ or $67$ if $x$ is stored in respectively big-endian
  and little-endian order.
\end{example}

Note that byte ordering does not affect the ordering of digits (or
bits) \emph{within} each byte.  This is because individual bits do not
have addresses; only bytes do.

While big-endian used to be dominant, most current machines use
little-endian byte order.  This can be confusing when inspecting raw
memory contents, as the convention in mathematical notation is
big-endian.  Fortunately, byte order is unimportant in most
programming tasks: it is handled by the programming language
implementation.  We can only observe the byte order when we explicitly
decompose values into their constituent bytes.  Bit-shifting integers
and similar always works as expected.

\subsection{Addresses and pointers}

The addresses used to index memory are unsigned numbers of a fixed
size that depends on the computer architecture.  32-bit addresses used
to be common, but most mainstream computers now use 64-bit addresses.
The size of the address is independent of how much memory is actually
physically installed in the computer---with 64 bits we can address a
memory comprising $2^{64}$ bytes, which is \emph{far} larger than any
current or planned computer.\footnote{As of this writing, the largest
  supercomputer in the world is the Japanese Fukagu, which contains
  approximately $2^{52}$ bytes of memory in total, although this is
  spread across 158976 distinct physical memories.}

As a model, we can imagine that addresses beyond the physical capacity
of the computer result in an error.  For example, in a computer with
$2^{30}B=1GiB$ of memory, addresses of $2^{30}$ and above would be
invalid.  The consequences of accessing an invalid address varies
depending on the machines, but in most cases the program will be
terminated---under Unix-like operating systems, this is called a
\emph{segmentation fault}.  You will see lots of these in your own C
programs.

To the computer, an address is merely an unsigned number, and
programming directly with them is notoriously error-prone.  Most
languages do not expose addresses at all, and even those that do tend
to give them a distinct type to avoid mistakes.  In C, we can use the
\texttt{\&} prefix operator to take the address of a variable
\texttt{x}, as follows:
\begin{center}
  \texttt{\&x}
\end{center}
If \texttt{x} has type \texttt{T}, then \texttt{\&x} has type
\texttt{T*}, read as ``pointer to \texttt{T}''.  In C, a variable that
contains an address is called a \emph{pointer}.  We can
\emph{dereference} a pointer by using the prefix operator \texttt{*}.
This means taking the value of the pointer (which is an address) and
retrieving the value at that address.  For example, if \texttt{px} has
type \texttt{T*}, the expression \texttt{*px} has type \texttt{T}.  Be
careful not to confuse the use of \texttt{*} in \emph{types} (where it
\emph{adds} a layer of indirection) with the use of \texttt{*} in
\emph{expressions} (where it \emph{removes} a layer of indirection).
Pointers are an inherently difficult topic, but flaws in C's syntax
makes it no easier.

If you print the addresses used for variables in your own programs,
you will see that many have values that far exceed the amount of
physical memory in your computer.  This is due to virtual memory,
which we will return to later, but means there is a decoupling between
the addresses seen by software and the actual physical addresses used
by hardware.  As a model, we can see program memory not as an array
with contiguous valid addresses, but instead as an array with ``gaps''
of invalid addresses.

\section{Text and Characters}
\label{sec:text}

Computers understand only bits, aggregated into bytes.  These are
stored as voltage differences inside electronic circuitry.  In order
for a computer to be useful, this data must be made comprehensible to
a human.

This usually happens through an \emph{input-output} (IO) device.
Imagine an electronic typewriter that receives bytes over a wire. Upon
receiving a byte, it consults a table called a \emph{coded character
  set} that maps each of the $256$ possible bytes to a character, and
then prints the corresponding character on a piece of paper. This was
exactly how early \emph{teletypes} worked, and the idea of associating
numbers with characters, which are then printed or shown as
appropriate, is still relevant. Today, it is unlikely that you
interact with your computer through a teletype, and instead the bytes
are passed through multiple layers of hardware and software until your
monitor ultimately illuminates multiple tiny LEDs to form the desired
shape on the screen.

\begin{definition}[Coded character set]
  A mapping from integers to characters.
\end{definition}

Many character sets used to proliferate, but \emph{ASCII} eventually
became dominant.  ASCII, shown on \cref{tab:ascii} defines only $127$
characters, leaving the 8th bit free.  Originally this 8th bit was
used to perform error correction when data was transmitted across
noisy telephone lanes, but later it was used to extend the ASCII-table
with language-specific variants.  For example, the ISO-8859-1
character set added support for most Western European characters.
Countries that did not use the Latin alphabet (such as Russia, China,
or Japan) defined their own non-ASCII character sets.  As always, data
was not self-describing, so in order to interpret a byte sequence as
text, you had to know which character set was used to encode it.

Eventually, the Unicode standard was established, which was conceived
as a single character set that could encompass \emph{all} the worlds
languages.  In ASCII and its variants, each byte corresponds to a
single character, but this is insufficient for Unicode, which as of
this writing defines 149,186 characters\footnote{For a very wide
  definition of ``character''---everything from hieroglyphs,
  right-to-left indications, combining accents, and emoji are part of
  Unicode.}.  While Unicode is still rooted in the idea of mapping
numbers to characters, it provides multiple ways of encoding each
number as (usually multiple) bytes.  The most common encoding, called
UTF-8, is a variable-width encoding where the number of bytes per
character depends on the character being encoded.  However, for the
subset of Unicode corresponding to ASCII, UTF-8 has the property that
any ASCII text is also valid UTF-8, and encodes the same characters.
For simplicity, we will stick to ASCII in the following.

\begin{table}
  \centering
   \tiny\ttfamily
  \begin{tabular}{|cc|cc||cc|cc|cc|cc|cc|cc|}
    \multicolumn{4}{l}{\textsf{\textbf{Control characters}}} & \multicolumn{12}{l}{\textsf{\textbf{Normal characters}}} \\
    \hline
    000 & nul & 016 & dle & 032 &  \textvisiblespace & 048 & 0 & 064 & @ & 080 & P       & 096 & \textquoteleft& 112 & p        \\
    001 & soh & 017 & dc1 & 033 &  !                 & 049 & 1 & 065 & A & 081 & Q       & 097 & a             & 113 & q        \\
    002 & stx & 018 & dc2 & 034 &  ``                & 050 & 2 & 066 & B & 082 & R       & 098 & b             & 114 & r        \\
    003 & etx & 019 & dc3 & 035 &  \#                & 051 & 3 & 067 & C & 083 & S       & 099 & c             & 115 & s        \\
    004 & eot & 020 & dc4 & 036 &  \$                & 052 & 4 & 068 & D & 084 & T       & 100 & d             & 116 & t        \\
    005 & enq & 021 & nak & 037 &  \%                & 053 & 5 & 069 & E & 085 & U       & 101 & e             & 117 & u        \\
    006 & ack & 022 & syn & 038 &  \&                & 054 & 6 & 070 & F & 086 & V       & 102 & f             & 118 & v        \\
    007 & bel & 023 & etb & 039 &  \textquotesingle  & 055 & 7 & 071 & G & 087 & W       & 103 & g             & 119 & w        \\
    008 & bs  & 024 & can & 040 &   (                & 056 & 8 & 072 & H & 088 & X       & 104 & h             & 120 & x        \\
   009 & tab & 025 & em  & 041 &    )                & 057 & 9 & 073 & I & 089 & Y        & 105 & i             & 121 & y       \\
   010 & lf  & 026 & eof & 042 &  *                 & 058 & : & 074 & J & 090 & Z        & 106 & j             & 122 & z       \\
   011 & vt  & 027 & esc & 043 &  +                 & 059 & ; & 075 & K & 091 & [        & 107 & k             & 123 & \char`\{\\
   012 & np  & 028 & fs  & 044 &  \textquoteright   & 060 & < & 076 & L & 092 & \char`   & 108 & l             & 124 & |       \\
   013 & cr  & 029 & gs  & 045 &  -                 & 061 & = & 077 & M & 093 & ]        & 109 & m             & 125 & \char`\}\\
   014 & so  & 030 & rs  & 046 &  .                 & 062 & > & 078 & N & 094 & \^{}     & 110 & n             & 126 & \~{}    \\
   015 & si  & 031 & us  & 047 &  /                 & 063 & ? & 079 & O & 095 & \char`\_ & 111 & o             & 127 & del   \\
    \hline
      \end{tabular}
      \caption{The ASCII table, mapping decimal numbers to the
        corresponding character.  Note that not all of these
        characters are intended for humans---some are control
        characters for operating the teletype or otherwise controlling
        the communication link.  While some of these are still used,
        mainly the ones corresponding to whitespace, most are rarely
        seen today.}
  \label{tab:ascii}
\end{table}

\subsection{Numbers and Text}

When we want to print a number, we have to produce a byte sequence
corresponding to an encoding of the digits.  For example, the number
\[
  1234_{10} = 04d2_{16}
\]
might be stored like this in memory, using decimal notation, 2 bytes,
and little-endian representation:
\[
  \begin{array}{|c|c|}
    \hline 210 & 4 \\\hline
  \end{array}
\]
We cannot merely submit these two bytes to the teletype.  The number
$210$ does not correspond to any ASCII character at all, and $4$
corresponds to the control character \texttt{eot} (for ``end of
transmission'').  Instead, we must convert it to the byte
string
\[
  \begin{array}{|c|c|c|c|}
    \hline 49 & 50 & 51 & 52 \\\hline
  \end{array}
\]
which we can then submit to the teletype or write to a text file.
Note that the number $49$ encodes the ASCII digit \texttt{1}, $50$
encodes \texttt{2}, etc.  It is an easy mistake to conflate the number
in memory with the (in principle arbitrary) number that identifies a
\emph{digit character} in the ASCII table.  Note also that endianness
business does not matter for human-readable text.  The order in which
the characters appear in memory (or the order in which they are
submitted to the IO device) are also the order in which they will
appear to a human.\footnote{There are character encodings that
  represent each characters using a fixed number of bytes, such as
  UTF-32 which uses four bytes per character.  These are subject to
  byte ordering issues \emph{within} each character.  Text using this
  encoding is usually preceded by a \emph{byte order mark} indicating
  the endianess used.  This is not a problem for UTF-8.  If you make
  correct choices in life, you will never have to worry about this.}

\subsection{Strings in C}

A sequence of characters is colloquially called a \emph{string} in
most programming languages, and is typically one of the most central
types.  Unfortunately, C has no built-in string type.  Instead C
represents strings with the type \texttt{char*}, representing the
address of the first character in the string, and indicates the end of
the string with a single byte with the value $0$.

This representation, called zero-termination, is recognised as one of
the greatest design mistakes in computer science and has been the
cause of countless security holes.  The reason is that it is awfully
easy to forget to add the trailing $0$ byte when constructing a
string.  When later code then attempts to work on the string, it will
go past the intended end and into arbitrary memory, until a $0$ byte
happens to be encountered by happenstance.

Most other languages employ a representation where a string is stored
alongside an integer containing the size of the string.  This is much
less error-prone.

%%% Local Variables:
%%% mode: latex
%%% TeX-master: "notes"
%%% End:
