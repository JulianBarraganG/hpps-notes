\chapter{Floating-Point Numbers}
\label{chap:floats}

Integers are all well and good, but we often need to solve problems
that require the use of fractions---that is, we need a machine
representation of some kind of approximation of rational or real
numbers.  In this section we will look at the various solutions one
can come up with to this problem, but ultimately we will focus in
detail on \emph{floating-point} numbers, which are the most common
implementation of fractional numbers in modern computers.  This is an
\emph{enormous} topic, and many devote their entire careers to
studying number representations.  This chapter is relatively
superficial, but the particularly interested student can continue
their studies by reading the \emph{Handbook of Floating-Point
  Arithmetic}~\cite{muller2018handbook}, from which most of the
information in this chapter is sourced, or taking courses on numerical
analysis.

\section{The Basic Problem}

Fractional numbers are inherently more difficult to implement on
computers because of their density.  Between the numbers $x$ and $y$,
there are $|x-y|$ integers.  Thus, when we define a fixed-size integer
representation, there will be a smallest and largest number, but there
will be no gaps inside this interval.  In contrast, between $x$ and
$y$ there is an \emph{infinite} number of rational numbers (assuming
$x\neq y$).  With a $w$-bit fixed-size representation we can at most
represent $2^{w}$ distinct values, so any nonempty interval of
rational numbers will necessarily have numbers we cannot represent
using a fixed-size encoding.  Number representations using an
arbitrary and value-dependent number of bits can be defined that can
represent any computable real numbers using a variable number of bits,
but these are implemented in software rather than in hardware, and are
\emph{much} slower than fixed-size representations.

One obvious representation is to represent fractions as pairs of
integers.  For example, the fraction
\[
  \frac{1}{3}
\]
can be straightforwardly represented as the pair $(1,3)$.  Using
$w$-bit integers, such a pair could be represented as a $2w$-bit
vector.  The main problem with this representation is that a given
number has many possible encodings---e.g. $(1,3)$ and $(2,6)$ both
represent the same number.  Fractions can be reduced using algorithms
that have been known for literally thousands of years, but these are
computationally expensive---certainly not something we want a computer
to do as part of routine arithmetic.  Another consequence of this
redundancy is that we also waste encoding space.  As much as possible,
we desire a representation where distinct bit patterns encode distinct
numbers.

\subsection{Precision and Accuracy}

Informally, the terms \emph{precision} and \emph{accuracy} are often
used interchangeably.  In the following we will use them with very
specific meanings.

\begin{description}
\item[Precision] is how many digits are present in a result.  As an
  analogy, if someone asks you for the current time and you answer
  ``twelve hours, four minutes, three seconds, fourteen miliseconds'',
  then your answer is very precise.
\item[Accuracy] is how close an approximated or rounded result is to
  the \emph{true} value.  Following on the analogy above, that very
  precise answer may be quite inaccurate if it's actually 8 in the
  morning.  Accuracy is also sometimes called \emph{exactness}.
\end{description}

Since any finite-sized representation of rational numbers will always
have gaps, the result of any computation must be rounded to the
nearest number that can actually be represented in limited precision.
This is a source of inaccuracy.  While we focus mostly on
representation rather than calculation, we will return to the issue of
rounding in \cref{sec:rounding}.

\section{Fixed-point Fractional Numbers}

In previous sections we saw that that when we write a binary integer
such as $10010101_{2}$ it is basically interpreted the same way as
$149_{10}$---the structure is the same, except that the digit weights
are powers of 2 instead of powers of 10.  This correspondence inspired
the unsigned integer representation of \cref{def:bits2n}.  Can we
perhaps use the same idea to represent fractional numbers?  Yes.  A
decimal fraction $123.456_{10}$ can be viewed as having weights that
are \emph{non-negative} powers of $10$ on the left of the decimal
point, and $negative$ powers to the right.  To compute its value, we
multiply each digit with the weight.

\begin{example}[Interpretation of decimal fraction $123.456_{10}$]
\[
  \begin{array}{lcccccccccccccc}
    \mathrm{Digit}   &1&&2&&3&.&4&&5&&6 \\
    \mathrm{Weight} & 10^{2} && 10^{1} && 10^{0} && 10^{-1} && 10^{-2} && 10^{-3} \\
    \mathrm{Sum} & 100 &+& 20 &+& 3 &+& \frac{4}{10} &+& \frac{5}{100} &+& \frac{6}{1000}
  \end{array}
\]
\end{example}
Similarly, a binary fractional number $1001.0101$ has weights that are
powers of $2$, and the ones to the right of the \emph{binary
  point}\footnote{Corresponding to the \emph{decimal point} in a
  base-10 system.} have negative exponents:
\begin{example}[Interpretation of binary fraction $1001.0101_{2}$]
\[
  \begin{array}{lcccccccccccccccc}
    \mathrm{Digit}   &1&&0&&0&&1&.&0&&1&&0&&1 \\
    \mathrm{Weight} & 2^{3} && 2^{2} && 2^{1} && 2^{0} && 2^{-1} && 2^{-2} && 2^{-3} && 2^{-4} \\                                                                          \mathrm{Sum} & 8 &+& 0 &+& 0 &+& 1 &+& 0 &+& \frac{1}{4} &+& 0 &+& \frac{1}{16} \\
    &\multicolumn{8}{l}{= 9.3125_{10}}
  \end{array}
\]
\end{example}

For some word size $w$, we must specify how many of the bits we
allocate to the integral part (before the binary point) and how many
are part of the fraction.  In the example above we used a symmetric
representation with $4$ bits devoted to the integral and fractional
part.  The crucial property is that this allocation is fixed for a
given format, and cannot vary.  This is why this number representation
is called \emph{fixed point}.

\begin{definition}[Fixed-point number]
  A number representation where a fixed number of bits is allocated to
  the fractional part.
\end{definition}

\begin{definition}[Bit vector interpreted as unsigned fixed point]
  \[
    \mathrm{Fix2Q}(\bitvector{x_{n+m-1}\cdots x_{m}~x_{m-1}\cdots x_{0}}) =
    \sum_{i=0}^{n+m-1} x \cdot 2^{i-m}
  \]
  where $n,m$ is the number of bits allocated to integral and
  fractional part, respectively.
  \label{def:fx2Q}
\end{definition}

For simplicity we deal only with non-negative numbers in this
representation.  Support for negative numbers can be implemented by
either adding a sign bit, or by interpreting the integral part as a
Two's Complement number.

One nice property of fixed-point representation is that we can divide
and multiply by bit-shifting, just as with integers.  Unfortunately,
it also has serious limitations.

\paragraph{Limitation \#1} This representation can only represent
numbers of the form
\[
  \frac{x}{2^{k}}
\]
for some $x,k$.  This means that numbers such as $0.3_{10}$ cannot be
represented exactly, but must be approximated.  Allowing 4 bits for
the fraction, the best approximation is
\begin{equation*}
  0.0101_{2} = 0.3125_{10}
\end{equation*}
This is not a restriction that is particular to our use of binary.
Any radix will have numbers that cannot be represented, just as we
cannot write
\[
  \frac{1}{3}
\]
with any finite decimal sequence.

\paragraph{Limitation \#2} Given $w$ bits to represent fixed-point
numbers, we need to decide once and for all how many bits we dedicate
to the integral part, and how many to the fraction.  If we allocate
many bits to the integral part, we will be able to represent larger
numbers, but the distance between neighbouring representable numbers
will be breater.  This is illustrated on \cref{fig:fixedpoint}.

Note in particular that fixed-point representations have lower
relative precision for numbers close to zero.  Suppose $w=8$ and we
are using $4$ bits for the fraction; then the next highest number from
1 is $1.00625$---a relative distance of $0.00625$.  But for $15$, the
next number is $15.00625$---a relative distance of $0.00004167$.

\begin{figure}
  \begin{framed}
  \centering
    \begin{description}
  \item[\textbf{$1$ bit for fraction}]\hfill
    \begin{itemize}
    \item Largest number: $1111111.1_{2} = 127.5_{10}$
    \item Increment: $0000000.1_{2}=0.5_{10}$
    \end{itemize}
  \item[\textbf{$7$ bits for fraction}]\hfill
    \begin{itemize}
    \item Largest number: $1.1111111_{2} = 1.9921875_{10}$
    \item Increment: $0.0000001_{2}=0.0078125_{10}$
    \end{itemize}
  \item[\textbf{$4$ bits for fraction}]\hfill
    \begin{itemize}
    \item Largest number: $1111.1111_{2} = 15.9375_{10}$
    \item Increment: $0000.0001_{2}=0.0625_{10}$
    \end{itemize}
  \end{description}
  \end{framed}
  \caption{The fixed-point dilemma for $w=8$.  The \emph{increment} is the
    distance between neighbouring numbers.}
  \label{fig:fixedpoint}
\end{figure}

Ideally, we want a number representation where the relative distance
between numbers is constant---meaning that the absolute distance
between representable numbers close to zero is small, while the
distance between numbers far from zero is large.  Another way of
looking at this of the $2^{w}$ distinct numbers possible for a $w$-bit
representation, most of them should be close to zero.  We can view
this as a number comprising $w$ digits where the binary point is not
fixed, but rather ``floating'', which is why the common name for this
type of numbers is \emph{floating-point numbers}.

\section{Intuition behind floating-point numbers}

Before delving into the technical details of floating-point numbers,
let us build up a little intuition for how they work.  floating-point
numbers (or just \emph{floats}) are inspired by decimal scientific
notation where a number is represented as
\begin{equation}
  x = \alpha \times 10^{\beta}
\end{equation}
where the \emph{significand} $\alpha$ is any number and the exponent
\emph{exponent} $\beta$ is an integer.  Obviously we can represent any
number simply by setting $\beta=0$ and using only $\alpha$.  However,
a common convention is to require that $|\alpha|<10$, in which case we
say a number is \emph{normalised}.

\begin{example}[Normalisation in scientific notation]
  The number
  \[
    x = -123 \times 10^{-1}
  \]
  is not normalised, but we can normalise it by dividing the mantissa
  by $10$ and incrementing the mantissa by $1$, giving
  \[
    x -1.23 \times 10^{1}.
  \]
\end{example}

Now suppose we make this representation finite by requiring
\[
  -100 < \beta < 100
\]
and also that $\alpha$ has exactly two digits to the right of the
decimal point, meaning it is always of the form $x.xx$, possibly with
a sign.  This means we can now encode any number as three digits for
$\alpha$ (and a sign) plus two digits for $\beta$ (and a sign).  But
what are the consequences for which numbers can be represented?

One consequence is that we now have a largest representable number,
namely the one where $\alpha=9.99$ and $\beta=99$.  Another is that
the distance between neighbouring numbers now depends on the exponent
$\beta$.
\begin{example}[Distances between neighbouring numbers]
\begin{align}
  |1.23\times 10^{1} - 1.24\times 10^{1}| &= 0.1 \\
  |9.99\times 10^{1} - 0.01\times 10^{2}| &= 0.1 \\
  |0.01\times 10^{1} - 0.02\times 10^{2}| &= 0.2 \\
  |1.23\times 10^{50} - 1.24\times 10^{50}| &= 5
\end{align}
\end{example}

This is the basic idea behind floating-point numbers, and while
floating-point number formats have significant extra complexities to
deal with various numerical challenges and to make them efficiently
implementable on binary electronic computers, it all comes back to
this idea of having a normalised scientific representation comprising
a significand and exponent.

\section{IEEE 754}

floating-point formats were widely supported even by the earliest
computers, but they diverged widely in how they were implemented, how
much precision they offered, and what semantics they used for
rounding.  This made it difficult for numerical programs written on
one computer to run correctly on another computer.  Significant effort
was spent on standardising floating-point arithmetic, and in 1985, the
IEEE 754 standard was ratified.  It has since been updated and
extended, but the core concepts and principles are unchanged.  The
IEEE 754 standard is now essentially universally supported, except for
very specialised processors.

Apart from specifying rules and encodings of conventional rational
numbers, IEEE 754 also mandates the existence of certain special
values:

\begin{itemize}
\item NaN (``not a number''), a family of special values that are
  produced by invalid operations such as $\sqrt{-1}$.  The purpose of
  NaN is to ensure that all arithmetic operations are ``closed'', in
  that they produce a value that is representable in the IEEE 754
  value representation, even though it is literally speaking \emph{not
    a number}.
\item Positive and negative infinity, which have two important uses.
  One is to represent the result of operations such as division by
  zero.  Another is to represent overflow, where the result of an
  arithmetic operation lies outside the representable number range.
  Compare this to integers, where the most typical result is
  wraparound.
\end{itemize}

IEEE defines various floating-point formats.  All of them are
characterised by four integers

\begin{itemize}
\item a \emph{radix}, which we will assume to be 2 although IEEE 754
  also specifies decimal floating-point formats;
\item a \emph{precision} $p \geq 2$, roughly corresponding to the
  number of ``significant bits'' in the significand;
\item two \emph{extremal exponents} $e_{\text{min}},e_{\text{max}}$
  such that $e_{\text{min}} < e_{\text{max}}$.  In all formats
  specified by IEEE 754, $e_{\text{min}}=1-e_{\text{max}}$.
\end{itemize}

The two most common IEEE 754 formats are binary32 and binary64,
corresponding to the \texttt{float} and \texttt{double} types in most
C-like programming languages.  The parameters for the most common
binary IEEE 754 formats are shown in \cref{tab:ieee-formats}.

\begin{table}
  \centering
  \begin{tabular}{|l||c|c|c|}
    \hline
    Name & binary16 & binary32 & binary64 \\\hline
    Informal name & Half precision & Single precision & Double precision \\\hline
    C type & N/A & \texttt{float} & \texttt{double} \\\hline
    $p$ & $11$ & $24$ & $53$ \\\hline
    $e_{\text{max}}$ & $+15$ & $+127$ & $+1023$ \\\hline
    $e_{\text{min}}$ & $-14$ & $-126$ & $-1022$ \\\hline
  \end{tabular}
  \caption{The most common IEEE 754 floating-point formats.  Half
    precision floats are not supported in standard C, but are
    relatively common in graphics programs.  More exotic non-standard
    formats are also used in machine learning.}
  \label{tab:ieee-formats}
\end{table}

A finite floating-point number expressed in such a format is a number
$x$ for which there exists a representation $(s,m,e)$ defined as
follows.

\begin{definition}[Representation of floating-point number $x$]
\[
  x = -1^{s} \cdot{} m \cdot{} 2^e
\]
where
\begin{itemize}
\item $s$ is the \emph{sign bit}, which is 1 for negative numbers;
\item $m$ is a number called the \emph{normal significand}.
\item $e$ is an integer such that
  $e_{\text{min}} \leq e \leq e_{\text{max}}$, called the
  \emph{exponent}.
\end{itemize}
\label{def:float-rep}
\end{definition}

The possible values of $m$ is related to $p$ in a way that we will
specify in \cref{sec:float-bits}.  \Cref{def:float-rep} does not
specify unique representation for every number.  Consider

\begin{equation}
  16 = -1^{0}\cdot{} 16\cdot{}2^{0} = -1^{0}\cdot{} 32\cdot{}2^{-1}
\end{equation}

In order to ensure a unique representation for a given number, we
require that floating-point numbers are normalised, by always choosing
the representation for which the exponent is smallest (but of course
not less than $e_{\text{min}}$).  This implies that $0 \leq m < 2$.

Further, whenever $1 \leq m < 2$, we say that $x$ is a \emph{normal}
number.  Intuitively we can see $m$ as indicating a number of the form
$1.xxx_{2}$, where $m$ constitutes the binary digits to the right of
the binary point.  These digits are sometimes called the
\emph{fraction}.

Otherwise, when $e=e_{\text{min}}$, we must also have that
$0 \leq m < 1$, and we call this case a \emph{subnormal} number.  For
subnormal numbers, the bit before the binary point must necessarily be
0.  That is, in this case $m$ indicates a number of the form
$0.xxx_{2}$.  This means that when we encode $m$ in a bit vector, we
can elide the bit before the binary point, because it can be inferred
from $e$, thus saving a a bit.

The special case of zero will be discussed later.

\subsection{Bit Vector Representation}
\label{sec:float-bits}

By \cref{def:float-rep}, a floating-point number is represented by
three numbers $s,e,m$.  As a bit vector, these are encoded as tree
fields $S$, $E$ and $T$.  Further, some bit patterns are reserved to
express NaN, infinity, and zero.

\begin{definition}[Fields of a floating-point number $x$]
  \[
    \bitvector{\underbrace{S_{0}}_{1~bit}~\underbrace{E_{r-1}\cdots{E_{0}}}_{r~\text{bits}}~\underbrace{T_{p-2}\cdots{}T_{0}}_{p-1~\text{bits}}}
  \]
  where $S$ encodes $s$ directly, $E$ encodes $e$, and $T$ encodes
  $m$.
  \label{def:float-fields}
\end{definition}

The $S$ field is one bit in length and directly corresponds to $s$.
The $E$ field consists of $r$ bits and the $T$ field consists of
$p-1$ bits, but still represents $p$ bits of information, because an
extra bit can be deduced from the context as discussed above. How
these fields are decoded as $e$ and $m$ depends on their exact values.
In the normal case, $E$ encodes $e$ as a \emph{biased} number, which
we briefly saw in \cref{sec:words}.  Specifically, $E$ encodes an
unsigned integer from which we subtract a \emph{bias} $b$ to obtain
the possibly negative $e$.  For all IEEE formats, $b=e_{\text{max}}$,
which lets $E$ encode integers ranging from $e_{\text{min}}$ to
$e_{\text{max}}$.

In total, decoding a floating point number represented as a bit vector
in the form from \cref{def:float-fields} involves the following cases.

\begin{itemize}
\item If $E = \bitvector{1\cdots{} 1}$ (a sequence of ones) and
  $T \neq \bitvector{0\cdots{} 0}$, then the bit vector represents
  not-a-number (NaN).\footnote{Note that \emph{multiple} NaNs exist,
    since $T$ can have multiple possible values.  This is sometimes
    called the NaN \emph{payload}, and can be used to store
    information about the origin of the NaN value.}
\item If $E = \bitvector{1\cdots{} 1}$ and
  $T = \bitvector{0\cdots{} 0}$ then the bit vector represents
  $\pm\infty$, with sign determined by $S$.
\item If $E = \bitvector{0\cdots{} 0}$ and
  $T = \bitvector{0\cdots{} 0}$, then the bit vector represents
  $\pm 0$, with sign determined by $S$.
\item If $E \neq \bitvector{0\cdots{} 0}$ and
  $E \neq \bitvector{1\cdots{} 1}$, then the bit vector represents the
  normal number
  \[
    (-1)^{s} \cdot m \cdot 2^e
  \]
  where
  \[
    \begin{array}{ccr}
      b &=& e_{\text{max}} \\
      s &=& \mathrm{Bits2N}(S) \\
      m &=& 1+\mathrm{Bits2N}(T) \cdot{} 2^{1-p} \\
      e &=& \mathrm{Bits2N}(E) - b
    \end{array}
  \]
\item If $E = \bitvector{0\cdots{} 0}$ and
  $T \neq \bitvector{1\cdots{} 1}$, then the bit vector represents the
  subnormal number
  \[
    (-1)^{s} \cdot m \cdot 2^{e_{\text{min}}}
  \]
  where
  \[
    \begin{array}{ccr}
      s &=& \mathrm{Bits2N}(S) \\
      m &=& \mathrm{Bits2N}(T) \cdot{} 2^{1-p} \\
    \end{array}
  \]
\end{itemize}

These cases form a conversion function.  For space reasons, we will
not write it down in that form.

\begin{example}[Interpreting a bit vector as a float]
  Using the binary16 format from \cref{tab:ieee-formats}, we interpret
  \[
    \bitvector{111010001010001}
  \]
  as a float.  First we split the bit vector into fields.
  \[
    \begin{array}{lll}
      S=\bitvector{1} & E=\bitvector{11010} & T=\bitvector{0001010001}
    \end{array}
  \]
  Inspecing $E$ and $T$, we see we are dealing with a normal number.  We then compute
  \[
    \begin{array}{ccr}
      b &=& 15 \\
      s &=& 1 \\
      m &=& 1 + 81 \cdot 2^{-10} \\
      e &=& 11
    \end{array}
  \]
  giving the final result
  \[
    (-1)^{s} \cdot m \cdot 2^{e} = -2210
  \]
\end{example}

\subsection{Interesting Properties}

Working with floating-point numbers is a big topic, but there's a
handful of simple properties that are useful to remember.

All floating point numbers can be arithmetically negated (multiplied
by negative one), simply by negating the sign bit.  This is in
contrast to Two's Complement integers, where the most negative number
cannot be negated.

In all IEEE floating-point formats, the range of $e$ is symmetric
around zero, e.g. \cref{tab:ieee-formats} shows that for binary32
(``single precision''), $e$ ranges from $-126$ to $+127$.  When $e<0$,
the magnitude of the represented number is less than $1$.  This means
that approximately \emph{half} of all representable numbers are in the
interval $[-1,1]$!  Therefore, while we can represent astronomically
large numbers with a magnitude of more than $2^{127}$, only a
relatively tiny portion of the encoding space is devoted to these, and
any arithmetic in this end of the number line is likely going to be
subject to severe rounding error.

\section{Rounding}
\label{sec:rounding}

Most rational numbers have no representation as floating point
numbers.  Therefore, in general the result of an arithmetic operation
cannot be represented exactly in any floating point format, but has to
be \emph{rounded} to the nearest representable value.  As an example,
although both $1$ and $10$ can be represented, the division
\[
  \frac{1}{10}
\]
cannot be exactly represented in any binary floating point format, and
instead has to be approximated.  In binary32, the closest
approximation is
\[
  13421773 \cdot 2^{-27} = 0.10000000149011612.
\]

In the first floating-point systems, the way results were rounded was
not always specified, and frequently varied among computers.  The IEEE
754 standard mandates the concept of \emph{correct rounding}, where
the result of a single operation is calculated with infinite
precision, and then rounded to a representable value using one of
several possible \emph{rounding modes}\footnote{The 2008 revision to
  IEEE 754 changed the term to \emph{rounding direction attribute},
  but \emph{rounding mode} is still in widest use by practitioners.}.

\begin{definition}[Correct rounding]
  When the computed result of an operation is the same as if the
  operation was done with infinite precision and unlimited range, then
  rounded according to the chosen rounding mode.
\end{definition}

Essentially, when an operation is correctly rounded, it means that the
result is as close to being mathematically correct as we can represent
within the floating-point format.  IEEE 754 guarantees correct
rounding for addition, subtraction, multiplication, division, and
square root, but not for the various transcendental functions:
logarithms, sines, cosines, etc.  The exactness of these functions can
vary between platforms.

\subsection{Rounding modes}

IEEE 754 specifies five rounding modes:

\begin{itemize}
\item Round towards $-\infty$, also called ``rounding downwards''.
  When rounding a number $x$, this produces the largest representable
  number smaller or equal to $x$.
\item Round towards $\infty$, also called ``rounding upwards''.
  When rounding a number $x$, this produces the smallest representable
  number greater or equal to $x$.
\item Round towards $0$, which rounds downwards when $x>0$ and
  otherwise rounds upwards.
\item Round to nearest, ties to even.  This rounds to the closest
  representable number, picking the even number in case of ties.  This
  is the default rounding mode in the vast majority of systems.
\item Round to nearest, ties away from zero.  Picks the representable
  number with the largest magnitude in case of ties.
\end{itemize}

In most cases we do not worry about rounding modes when programming.
The default behaves sensibly in most cases, and changing the rounding
mode is usually a global change for the entire process, which makes it
difficult to use in a modular way.

\section{Arithmetic}
\label{sec:float-arithmetic}

While a full discussion on implementing floating-point arithmetic is
well outside the scope of this text, we are going to briefly discuss
how multiplication and addition is implemented.  For clarity of
exposition, we are going to assume that the operands are proper
numbers expressed as
\[
  (-1)^{s} \cdot m \cdot 2^{e}.
\]
In particular, they are not infinities or NaN (but may be subnormal).
In general, when an operation has a NaN operand, the result is also a
NaN.  Infinities are produced by overflow, and most operations
likewise preserve infinity, e.g:
\begin{equation}
  1 + \infty = \infty
\end{equation}
But some operations involving infinity may also produce NaN:
\begin{equation}
  \infty - \infty = \text{NaN}
\end{equation}

\begin{definition}{Floating-point overflow}
  When the magnitude of a floating-point number becomes too large to
  be representable and the result becomes $\pm\infty$.
\end{definition}

\begin{definition}{Floating-point underflow}
  When the magnitude of a floating-point number becomes too small be
  representable and the result becomes $0$.
\end{definition}

Note that overflow also occurs when a number becomes \emph{too
  negative}.  This is in contrast to integers, where this situation is
called \emph{integer underflow}.  With floating-point numbers,
underflow means that the number comes too close to zero, and ends up
being rounded to zero.

\subsection{Multiplication}

Due to the exponential representation, multiplication of
floating-point numbers is actually simpler than addition.  The product
of two floating-point numbers
\[
  (-1)^{s_{1}} \cdot m_{1} \cdot 2^{e_{1}}
\]
and
\[
  (-1)^{s_{2}} \cdot m_{2} \cdot 2^{e_{2}}
\]
is given by
\begin{equation}
  (-1)^{s_{3}} \cdot m_{3} \cdot 2^{e_{3}} = ((-1)^{s_{1}} \cdot m_{1} \cdot 2^{e_{1}}) \cdot ((-1)^{s_{2}} \cdot m_{2} \cdot 2^{e_{2}})
\end{equation}
  where
  \begin{align}
    s_{3} &= s_{1} \oplus s_{2} \\
    m_{3} &= m_{1} \cdot m_{2} \\
    e_{3} &= e_{1} + e_{2}
  \end{align}

  This might produce an un-normalised or unrepresentable number.  If
  $m_{e}\geq{}2$, meaning that the number is not normalised, we shift
  $m_{3}$ right (equivalent to repeatedly dividing by two) and
  increment $e_{3}$ (equivalent to doubling).  Then, if $e_{3}$ is out
  of range (either less than $e_{\text{min}}$ or greater than
  $e_{\text{max}}$), the operation has overflowed, producing
  $\pm\infty$.  Finally, we round $m_{3}$ to $p$ bits.  The addition
  of the exponents is just integer addition, which is straightforward.
  Implementing the multiplication of the significands is more
  complicated, but fortunately outside the scope of this text.

\subsection{Addition}

Given two floating-point numbers
\[
  x_{1} = (-1)^{s_{1}} \cdot m_{1} \cdot 2^{e_{1}}
\]
and
\[
  x_{2} = (-1)^{s_{2}} \cdot m_{2} \cdot 2^{e_{2}}
\]
we seek to compute the sum
\begin{equation}
  ((-1)^{s_{3}} \cdot m_{3} \cdot 2^{e_{3}}) = x_{1} + x_{2}
\end{equation}
meaning we have to find $s_{3},m_{3},e_{3}$.

First, assume without loss of generality that
$e_{1} \geq e_{2}$\footnote{We can always just flip the operands to
  the addition.}.  We rewrite $x_{2}$ such that its exponent becomes
$e_{1}$, but without changing its numerical value by also constructing
a new significand $m_{2}'$:
\begin{equation}
  x_{2}' = ((-1)^{s_{2}} \cdot m_{2}' \cdot 2^{e_{1}}
\end{equation}
Note that $x_{2}'$ is most likely not a normal representation, but
this is fine, as it is merely an intermediate result.  Now we can compute
\begin{align}
  s_{3} &= s_{1} \oplus s_{2} \\
  m_{3} &= m_{1} + m_{2}'
\end{align}
As with multiplication, this might not yield a normalised or
representable number.  We fix it a similar way. If $m_{3}\geq{}2$, we
shift $m_{3}$ right and increment $e_{3}$.  If $m_{3}<1$, we shift $m$
left and decrement $e_{3}$.  If $e_{3}$ is outside the range
$[e_{\text{min}},e_{\text{max}}]$, the result is $\pm\infty$.
Otherwise, we round $m_{3}$ to $p$ bits.

\begin{example}[Addition with $p=3$]
  \[
    \begin{array}{rlr}
      & (-1.01 \cdot 2^{2}) + (1.1 \cdot 2^{4}) \\
      =& (-1.01 \cdot 2^{2}) + (110.0 \cdot 2^{2}) & \text{Align exponents} \\
      =& (-1.01+110.0) \cdot 2^{2} & \text{Distributivity} \\
      =& 100.11 \cdot 2^{2} & \text{Add significands} \\
      =& 1.0011 \cdot 2^{4} & \text{Normalise} \\
      =& 1.01 \cdot 2^{4} & \text{Perform rounding} \\
    \end{array}
  \]
\end{example}

%%% Local Variables:
%%% mode: latex
%%% TeX-master: "notes"
%%% End:
