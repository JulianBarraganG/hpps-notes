% General stuff.  The idea is that this preample can also be used to
% compile each chapter separately.

\usepackage[utf8]{inputenc}

\usepackage{ntheorem}
\usepackage{amsmath}[ntheorem]
\usepackage{mathtools}
\usepackage{amsfonts}

\newcommand{\defeq}{\vcentcolon=}
\newcommand\lxor{\oplus}
\DeclareMathOperator{\BitOdot}{\odot^{\langle\rangle}}
\DeclareMathOperator{\BitAdd}{+^{\langle\rangle}}
\DeclareMathOperator{\BitMul}{\times^{\langle\rangle}}

\usepackage{caption}
\usepackage{subcaption}
\usepackage{graphicx}
\usepackage{todonotes}
\usepackage{url}
\usepackage{hyperref}

\usepackage{cleveref}

\usepackage{couriers}
\usepackage{listings}
\lstset{basicstyle=\ttfamily}

% Teach cleveref about listings.
\crefname{lstlisting}{listing}{listings}
\Crefname{lstlisting}{Listing}{Listings}

\newtheorem{definition}{Definition}[chapter]
\newtheorem{theorem}{Theorem}[chapter]
\newtheorem{corollary}[theorem]{Corollary}
\newtheorem{example}{Example}[chapter]

\setsecnumdepth{subsection}

%%% Local Variables:
%%% mode: latex
%%% TeX-master: "notes"
%%% End:
